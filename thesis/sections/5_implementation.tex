%% ==============================================
%%               IMPLEMENTATION
%% ==============================================
%% Author: Fabian Sorn
%% ==============================================

\chapter{Implementation of a Benchmark Framework}

\label{ch:implementation}

This chapter will focus on the actual implementation of the framework based on
the analysis and design from chapter \ref{ch:analysisAndDesign} and the use
cases from chapter \ref{ch:usecases}. As a name for our framework we choose
\emph{widgetmark}, a composite of the words \emph{widget} and \emph{benchmark}.
The language we use is for the implementation is Python 3, which gives us and
the framework's users access to newer language features like type hinting. We
will begin with an introduction into our project layout, followed by a more
detailed look at each sub-package and module of the project.




%% ==============================================
%%              Project Structure
%% ==============================================

\section{Project Structure}

The first part of the implementation is setting up our project, whose structure
can be seen in figure \ref{fig:application:implementation:structure}. The
project will follow python best practices with a package \emph{widgetmark} for
the source code and \emph{tests} for unit tests. Additionally the project
contains the two generated files \emph{\_version.py} and \emph{versioneer.py}
for getting version information from version control tags, \emph{README.md} for
an introduction into the project, \emph{setup.cfg} for tooling configuration and
\emph{setup.py} for dependency definition and installation instructions for
\inlinecode{Python}{setuptools}.

\begin{figure}[h]
    \centering
    \framebox[\textwidth]{%
        \begin{minipage}{0.9\textwidth}
            \dirtree{%
                .1 project.
                .2 widgetmark.
                    .3 base.
                        .4 benchmark.py.
                        .4 executor.py.
                        .4 launcher.py.
                    .3 qt.
                        .4 benchmark.py.
                        .4 executor.py.
                        .4 plot.py.
                        .4 util.py.
                    .3 cli.
                        .4 \_\_main\_\_.py.
                        .4 cli.py.
                        .4 cli\_view.py.
                        .4 loader.py.
                    .3 \_version.py.
                .2 tests.
                    .3 \_\_init\_\_.py.
                    .3 \dots.
                .2 setup.py.
                .2 setup.cfg.
                .2 versioneer.py.
                .2 README.md.
            }
        \end{minipage}
    }
    \caption{Widgetmark project structure}
    \label{fig:application:implementation:structure}
\end{figure}




%% ==============================================
%%             widgetmark.base
%% ==============================================

\section{Subpackage widgetmark.base}

The first sub-package of our implementation is the package
\inlinecode{Python}{widgetmark.base}. In this package, we will define all base
classes for our benchmarking framework, which are not part of the user
interface. To keep the framework extensible and not limited to Qt as its
\gls{gui} framework, we will not yet use Qt specific \glspl{api} in this
package. Qt specific implementations will be part of the
\inlinecode{Python}{widgetmark.qt} package.

\subsection{benchmark.py}

The first class for our implementation is the use case interface. To define an
interface in Python, we can create an abstract base class without providing an
implementation for our functions.

\lstinputlisting[ 
    language=Python,
    linerange={2-2, 14-14}
]{resources/widgetmark/base/benchmark.py}

We distinguish between three different types of information that can be defined
in a use case: mandatory functions like widget setup and operation, mandatory
attributes like performance requirements and optional attributes like a timeout.
Mandatory functions can be simply implemented as abstract methods using the
\inlinecode{Python}{abstractmethod} decorator provided by the
\inlinecode{Python}{abc} package.

\lstinputlisting[ 
    language=Python,
    linerange={116-117, 124-124}
]{resources/widgetmark/base/benchmark.py}

Non-mandatory attributes can be easily implemented as simple class attributes
with default values in the use case base class.

\lstinputlisting[ 
    language=Python,
    linerange={30-30}
]{resources/widgetmark/base/benchmark.py}

For implementing mandatory class attributes, \inlinecode{Python}{abc} does not
offer any standard decorators, but we can use \inlinecode{Python}{property} in
combination with the \inlinecode{Python}{abstractmethod} decorator to create
abstract properties for our class. While such an implementation does not
depict a mandatory attribute on a class level, the abstract property
can be implemented as a class attribute in subclasses.

\lstinputlisting[ 
    language=Python,
    linerange={56-58, 64-64}
]{resources/widgetmark/base/benchmark.py}

\lstinputlisting[ 
    language=Python,
    linerange={19-19, 22-22}
]{resources/widgetmark/benchLabel.py}

With this implementation, there is no visual difference for the user between
defining a mandatory or optional attribute. Both, class attributes and abstract
properties can also be accessed the same way after the class is initialized
using \inlinecode{Python}{instance.attribute_name}.

The next class defined in this module is the \inlinecode{Python}{UseCaseResult}
class, which is based on a single use case. For parameterized use cases, there
will be a result instance for each parameter combination. The result object
should comprise all the information we have recorded from the execution. To make
sure that use cases are read-only after their initialization, we hold the
information in private instance attributes and allow reading access only over
properties without defining property setters. Additionally, the class offers
convenience properties for later checks.  An example is a
\inlinecode{Python}{failed} property that can be used to check if exceptions
were raised during the execution.

\lstinputlisting[ 
    language=Python,
    linerange={173-180, 194-198, 245-246, 248-248}
]{resources/widgetmark/base/benchmark.py}

The last class defined in the module is the benchmarking window, which
incorporates the widget defined in the use case and grants access to the
operation we want to benchmark. Wrapping the operation allows us to access it
later in the executor without having a direct reference to the use case
instance itself. Since our goal is to keep the base implementation free from
\gls{gui} framework-specific \gls{api}, we won't subclass any Qt window but
only define common operations for windows.

\subsection{executor.py}

The executor module contains the class
\inlinecode{Python}{AbstractBaseExecutor}, which is responsible for executing
the benchmarking operations on the window, record the results and create a
fitting instance of the \inlinecode{Python}{UseCaseResult} class.

As \ref{fig:application:design:executor} shows, for recording execution times,
we need a loop like execution of the same operation over and over again. Simply
using a loop can lead to unrealistic results, since we have to give back the
control to the event loop in between operations to not skip on pending work
caused by our operation. The practical implementation of this is dependent on
the used \gls{gui} framework and will not be part of this class. Similar to the
benchmarking window, this implementation will be done in a fitting executor
subclass in package \inlinecode{Python}{widgetmark.qt}. As seen in
\ref{fig:application:design:executor} an executor instance is initialized for
every parameter combination of every use case. After the initialization, a
window can be attached to the executor through
\inlinecode{Python}{set_window()}. The execution can then be launched using the
\inlinecode{Python}{launch()} method. 

\lstinputlisting[
    language=Python,
    linerange={38-56}
]{resources/widgetmark/base/executor.py}

Internally the abstract protected function \inlinecode{Python}{_launch()} is
called, which gives \gls{gui} specific subclasses an opportunity to set up the
actual execution loop. This loop can use the protected function
\inlinecode{Python}{_redraw_operation}, which is responsible for calling the
operation on the window, collect timing information and control the profiler.

\lstinputlisting[
    language=Python,
    linerange={107-121}
]{resources/widgetmark/base/executor.py}

To get the current timestamp we can use the \inlinecode{Python}{time()}
function of the python package \inlinecode{Python}{time}. From this timestamp
and the last recorded one, we can calculate the delta timing for the last step.
The recorded timestamp will be saved in an instance attribute for the next loop
step as the last timestamp.

\lstinputlisting[
    language=Python,
    linerange={154-159}
]{resources/widgetmark/base/executor.py}

The protected function \inlinecode{Python}{_profile} is responsible for
controlling the profiler. Before executing the operation the profiler will be
activated. After the execution, the current profile will be disabled and the
stats are added to the previously saved ones.

\lstinputlisting[
    language=Python,
    linerange={124-137}
]{resources/widgetmark/base/executor.py}

The last important function is \inlinecode{Python}{_check_if_completed()}, 
which stops execution if either the timeout or the repeat counter of
the use case is reached. If one is the case, the protected function
\inlinecode{Python}{_complete()} is called. Since reaching the end of the
execution cycle means stopping the execution loop, this step is \gls{gui}
framework-specific and will be implemented in the subclasses. 

\lstinputlisting[
    language=Python,
    linerange={139-152}
]{resources/widgetmark/base/executor.py}

\subsection{launcher.py}

The module \inlinecode{Python}{launcher} contains the class
\inlinecode{Python}{Launcher}. The launcher can be initialized using either a
list of types, which are derived from the \inlinecode{Python}{UseCase} class or
with a list of file locations, where these classes are defined. If a list of
files is passed, the modules are imported using the
\inlinecode{Python}{importlib} package. To make sure we import only classes that
define use cases, we filter the found classes by their type. If the
user passes a specific use case name, the found types are additionally filtered
by their names.

\lstinputlisting[
    language=Python,
    linerange={123-126, 133-154}
]{resources/widgetmark/base/launcher.py}

The initialized launcher instance can be started using the
\inlinecode{Python}{run()} method, which takes a parameter
\inlinecode{Python}{profile} for controlling if the executor will create a
profile of the use case or not. Before initializing window and executor we
create a list of all parameter combinations. For each of these 
combinations, we initialize the use case class. To grant access in the use case
class to these parameters, we set them as instance attributes to the 
use case instance. If the use case class defines ,for example, the parameter
\inlinecode{Python}{'number': [0, 1]}, we can access the current value of this
parameter in the use case during execution by calling 
\inlinecode{Python}{self.number}.

\lstinputlisting[
    language=Python,
    linerange={90-119}
]{resources/widgetmark/base/launcher.py}

To resolve the right window and executor implementation for the backend defined
by the use case, we have the class \inlinecode{Python}{BackendResolver} which
iterates through a classes subclasses and compares their backend to the backend
requested in the use case.




%% ==============================================
%%             widgetmark.qt
%% ==============================================

\section{Subpackage widgetmark.qt}

This package contains qt specific implementations of the benchmarking window and
the executor, as well as the plotting abstraction layer.

\subsection{benchmark.py}

This module provides a Qt-based implementation of the
\inlinecode{Python}{AbstractBenchmarkingWindow} based on 
\inlinecode{Python}{QtWidgets.QMainWindow}. A function worth noting here is
\inlinecode{Python}{make_qt_abc_meta()}. It is responsible for resolving
metaclass conflicts between Qt classes and abstract classes by returning a
common metaclass for both.

\lstinputlisting[
    language=Python,
    linerange={7-9}
]{resources/widgetmark/qt/benchmark.py}

\subsection{executor.py}

This module provides a Qt-based implementation of the
\inlinecode{Python}{AbstractBenchmarkExecutor} based on
\inlinecode{Python}{QtCore.QObject}. For the execution loop, we will make use of
the \inlinecode{Python}{QtCore.QTimer} class with a timeout of zero seconds,
which will timeout every time it gets the opportunity to. This way we can make
sure that control is given back to the event loop each time the use case
operation was executed.
\cite{QTimer}

To the timer's timeout signal we can connect the executor's base's
\inlinecode{Python}{_redraw_operation} function, which handles profiling,
timing, and operation execution. After the timer is set up we can start the
event loop of our Qt application.

\lstinputlisting[
    language=Python,
    linerange={60-64}
]{resources/widgetmark/qt/executor.py}

After reaching the repeat counter or the timeout, we can use the
\inlinecode{Python}{stop()} function of the \inlinecode{Python}{QtCore.QTimer}
to stop the execution loop.

\lstinputlisting[
    language=Python,
    linerange={66-77}
]{resources/widgetmark/qt/executor.py}

The Executor is also responsible to start and quit the
\inlinecode{Python}{QtWidgets.QApplication}. One caveat worth mentioning in this
context is, that there should only be a single instance of the application
running, since multiple \inlinecode{Python}{QApplication} instances will lead to
problems. Because of this, the executor class has to make sure, that there is
no existing one before creating a new one.

\lstinputlisting[
    language=Python,
    linerange={79-87}
]{resources/widgetmark/qt/executor.py}

Since the window and the widget is exposed during the benchmark execution, the
user can interact with it. To not create additional work next to the use
case, we have to filter out all use case unrelated operations. As described in
section \ref{sec:fundamentals:qt:eventloop}, we can use event filters to filter
out unwanted mouse interactions.

\lstinputlisting[
    language=Python,
    linerange={9-10, 15-27, 29-32}
]{resources/widgetmark/qt/executor.py}

\subsection{plot.py}

The module \inlinecode{Python}{widgetmark.qt.plot} contains the implementation
of the abstraction layer for plotting operations. This abstraction layer should
allow us to perform common actions on different plotting libraries without
relying on their library-specific \gls{api}. The abstraction layer itself is
implemented as an abstract class, that provides a factory method that returns
the fitting subclass to the passed plotting library.

\lstinputlisting[
    language=Python,
    linerange={44-48, 54-57, 60-63, 70-76, 87-97}
]{resources/widgetmark/qt/plot.py}

Each operation we want to use will get its own method in this abstract class. As
an example we will have a look at the \inlinecode{Python}{add_item} function,
which allows us to add a new item to the plot, which can display some data.

\lstinputlisting[
    language=Python,
    linerange={107-108, 118-118}
]{resources/widgetmark/qt/plot.py}

For PyQtGraph we can implement the \inlinecode{Python}{add_item} function based on the \inlinecode{Python}{PlotDataItem}.

\lstinputlisting[
    language=Python,
    linerange={186-187, 189-196, 201-209}
]{resources/widgetmark/qt/plot.py}

For Matplotlib, on the other hand, we can use the \inlinecode{Python}{Line2D}
object for implementing the same function.

\lstinputlisting[
    language=Python,
    linerange={247-263, 279-289}
]{resources/widgetmark/qt/plot.py}

Both functions will return objects derived from the same
\inlinecode{Python}{AbstractDataItem}, which allows reading, writing and adding
data in the form of a two-dimensional NumPy array containing x and y-values. In our
use cases, we can add the plotting library we want to benchmark as a parameter
and pass it to the factory function \inlinecode{Python}{using()} of the
\inlinecode{Python}{AbstractBasePlot} class. This way we can benchmark the 
same use case in two completely different plotting libraries without having to
define two separate use cases.




%% ==============================================
%%             widgetmark.cli
%% ==============================================

\section{Subpackage widgetmark.cli}

The last package of our project's source code contains all modules related to
the \gls{cli}. It's purpose is accepting user input to start the launcher as
well as presenting the results. Additionally it saves the recorded profiling
statistics to separate files and starts the visualization using
\inlinecode{Python}{snakeviz}. For defining command line arguments, the python
package \inlinecode{Python}{argsparse} is used, which will automatically add a
help option, which explains the \gls{cli}'s usage when the user executes
\inlinecode{bash}{widgetmark --help}.

When installing \inlinecode{Python}{widgetmark} using
\inlinecode{Python}{setuptools}, we can register our \gls{cli} as a console
script. This allows us to make a Python function accessible from the command
line with a specific name. In the module \inlinecode{Python}{main.py}, we have
our central main method, which initializes the \gls{cli} and executes it. We
will register this one with the package's name \emph{widgetmark}.

\lstinputlisting[
    language=Python,
    linerange={4-6}
]{resources/widgetmark/cli/__main__.py}

\lstinputlisting[
    language=Python,
    linerange={65-66, 77-79, 99-99}
]{resources/widgetmark/setup.py}


