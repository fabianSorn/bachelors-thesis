%% LaTeX2e class for student theses
%% sections/abstract_de.tex
%%
%% Karlsruhe University of Applied Sciences
%% Faculty of  Computer Science and Business Information Systems
%% Distributed Systems (vsys)
%%
%% Prof. Dr. Christian Zirpins
%% christian.zirpins@hs-karlsruhe.de
%%
%%
%% Version 0.2, 2017-11-15
%%
%% --------------------------------------------------------
%% | Derived from sdqthesis by Erik Burger burger@kit.edu |
%% --------------------------------------------------------


\Abstract

Eine gute Performance spielt eine wichtige Rolle, wenn es um die Benutzbarkeit
von Software Anwendungen geht. Dies gilt im Konsumenten Bereich im selben Maße
wie für professionelle Anwendungen, wie beispielsweise Monitoring Software am
CERN, der europäischen Organisation für Kernforschung. Sie erlaubt es Experten,
eine Vielzahl an Komponenten zu überwachen und deren Einstellungen, wenn nötig,
anzupassen.

Eine essenzielle Komponente dieser Monitoring Anwendungen sind Graphen, die es
erlauben, eine große Datenmenge gut verständlich und kompakt zu präsentieren.
Starke Abweichungen oder generelle Trends in einem Datensatz können auf diese
Weise schnell erkannt und entsprechend reagiert werden. Die Arbeit wird in der
Applications Sektion der Beams Controls Gruppe erstellt. Mit dem Wechsel zu
PyQt, als Framework für neu entwickelte, native \gls{gui} Anwendungen stellt
sich die Frage, welche Python Graphen Bibliothek diese Aufgabe am besten
erfüllt. Neben dem Angebot an Features spielt die Geschwindigkeit in der
späteren Anwendung eine wichtige Rolle. Ist diese schlecht, wird Produktivität
des Nutzers mit der Software maßgeblich geschmälert, da dieser durch lange
Reaktionszeiten in seiner Tätigkeit ausgebremst wird.

Ziel dieser Arbeit ist die Konzeption und Entwicklung einer Lösung in Form eines
Frameworks, die es erlaubt, mehrere Graphen Bibliotheken auf ihre Performanz in
bestimmten Nutzungsszenarien hin zu untersuchen. Der Nutzer sollte dabei seine
eigenen Szenarien ohne großen Aufwand frei abbilden können, ohne spezifische
Kenntnisse über Performance Tests haben zu müssen. Die Nutzung sollte dabei
vertraut für Nutzer anderer Python Test Frameworks sein und einen schnellen
Einstieg erlauben. Durch auf Wunsch erstellte Profile soll es geübten Nutzern
der jeweiligen Bibliothek ermöglicht werden, Flaschenhälse zu finden und diese
bei Bedarf zu verbessern.

Um dieses Ziel zu erreichen, werden nach der Darstellung der grundlegenden
Themenbereiche wie Datenvisualisierung, Benchmarking und dem Anwendungsframework
Qt, Nutzungsszenarien präsentiert, die aus echten Monitoring Anwendungen
herrühren, die während der Erstellung dieser Arbeit am CERN entwickelt werden.
Basierend auf deren Ansprüchen und den zuvor gesammelten Information wird ein
Konzept für die Funktionsweise zentraler Aufgaben und den Aufbau involvierter
Komponenten entwickelt. Die Implementierung dieses Konzeptes erfolgt in der
Programmiersprache Python in der Version 3. Die Überprüfung der Implementierung
erfolgt in zwei Schritten. Zu Beginn werden alle gesammelten Nutzungsszenarien
mithilfe des Frameworks implementiert und ausgeführt, um sicherzustellen, dass
die Implementierung den Ansprüchen genügt. Darauf folgt eine Präsentation der
erzielten Ergebnisse. Zur Evaluierung der Genauigkeit der gemessenen Ergebnisse,
wird ein weiteres Szenario mit einer Implementierung in Form einer echten
minimalen Anwendung verglichen.
