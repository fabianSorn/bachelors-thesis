%% LaTeX2e class for student theses
%% sections/conclusion.tex
%%
%% Karlsruhe University of Applied Sciences
%% Faculty of  Computer Science and Business Information Systems
%% Distributed Systems (vsys)
%%
%% Prof. Dr. Christian Zirpins
%% christian.zirpins@hs-karlsruhe.de
%%
%%
%% Version 0.2, 2017-11-15
%%
%% --------------------------------------------------------
%% | Derived from sdqthesis by Erik Burger burger@kit.edu |
%% --------------------------------------------------------


\chapter{Conclusion}
\label{ch:conclusion}

This chapter will provide a summary over this work as well as a outlook in the
future for further improvements.

\section{Summary}
\label{sec:Conclusion:Summary}

The purpose of this work was to analyze the performance requirements for python
plotting libraries at CERN as well as the design and implementation of a
solution which allows use case driven performance benchmarking.  In the
beginning we had a look on all major technologies and topics involved in the
later solution to create a common knowledge base for the following chapters.
Following that, we investigated use cases for plotting libraries at CERN from
which we were able to derive the most important metrics to describe and analyze
performance. In the last part we focused on the design and development of a
solution which is capable of executing the prior collected use cases and return
a meaningful result describing the performance in the given use case. The
accuracy of the produced results was compared to an example application
implementation which is capable of measuring the same performance metrics.

\section{Outlook}
\label{sec:Conclusion:Outlook}

Based on the developed \emph{widgetmark} framework, users can conveniently
create own use cases for their own plotting needs and easily measure the
performance in reached in different plotting libraries. This allows them to
choose a fitting plotting library not only based on the offered features but
also on the performance capabilities they need.

While its use at this point is relatively narrow, there are multiple
opportunities to extend the purpose of the framework. For other Gui Frameworks,
new backends could be implemented, for more plotting libraries the plotting
abstraction layer can be extended and for other use case scenarios, the use case
interface could be extended for example with a setup and tear down phase.
To keep widgetmark as modular as possible, it could be extended by plug-in
functionalities, which would allow extending the framework more easily without
having to change the base implementation.

Another area of improvement is minimizing the overhead added by the framework
during the execution. By minimizing this overhead the results could be even
closer to the later real life performances.

Additionally the findings in analysis and design can bring insights into other
areas of software performance evaluation and can be a template for other
benchmarking frameworks which do not focus on \gls{gui} widgets.
