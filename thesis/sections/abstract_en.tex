%% LaTeX2e class for student theses
%% sections/abstract_en.tex
%%
%% Karlsruhe University of Applied Sciences
%% Faculty of  Computer Science and Business Information Systems
%% Distributed Systems (vsys)
%%
%% Prof. Dr. Christian Zirpins
%% christian.zirpins@hs-karlsruhe.de
%%
%%
%% Version 0.2, 2017-11-15
%%
%% --------------------------------------------------------
%% | Derived from sdqthesis by Erik Burger burger@kit.edu |
%% --------------------------------------------------------


\Abstract
Good performance plays an essential role in the usability of software products.
This is not only true for consumer software but also for professional software
like monitoring GUI applications used at CERN, the European Organization for
Nuclear Research. They allow experts to easily monitor a large number of devices
and adapt their settings when necessary.

An essential component of such monitoring applications are graphs, which allow
the presentation of a large data set in an easily understandable and compact
form factor. Strong deviations and general trends can be discovered more easily
which allows reacting to them. The work will be conducted in the Applications
section in the Beams Controls group. With the change to PyQt for new native GUI
applications, the question raises, which python library is suitable for this
task. Next to the features of each library, their speed in the later application
plays a central role. A bad performance will significantly lower the
productivity of the application's user, since he is slowed down by having to wait
for the system to respond to his interaction.

Goal of this work is to develop a framework, which allows testing the
performance of python graph libraries in specific use cases. The user has to be
possible to depict his use cases using the framework without much effort and
without having to be familiar with performance testing. Using the framework
should be easy for user's familiar with other python testing frameworks. Through
the on demand creation of profiles for the given use case, advanced users should
have the opportunity to find performance bottle-necks and eliminate them.

To reach this goal we will begin with a general overview about the fundamental
topics of data visualization and benchmarking as well as the application
framework Qt. Following that, real use cases are presented which originate from
monitoring applications, which are being developed at CERN at the time of
writing. Based on these requirements and prior findings a concept for the
central tasks of the framework as well as all involved components is developed.
The implementation will be done using python 3 as the programming language of
choice. The evaluation of the implementation will be done in two separate steps.
First all collected use cases will be implemented using the framework to make
sure that the functionality will meet our requirements to make sure that the
functionality will meet our requirements. This is followed by the presentation
of the recorded results for each use case. To evaluate the accuracy of the
recorded results, a separate use case will be compared to a minimal
implementation of the fitting application.
